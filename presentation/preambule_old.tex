%%Ceci est un testaa
\usepackage[utf8]{inputenc}
\usepackage{pifont}
\usepackage{braket}
\usetheme{Frankfurt}
% \usecolortheme{wolverine}
\setbeamercolor{structure}{fg=magenta!80!white}
%\setbeamertemplate{background canvas}[vertical shading][top=white,bottom=orange]
\usepackage[french]{babel}
\usepackage{mathrsfs}
\usepackage{stmaryrd}
\usepackage{transparent}
%\usepackage{mathenv}
\usepackage{url}
\usepackage{graphicx}
\usepackage{cite}
\usepackage{tikz}
\usepackage{calc}
\usepackage{bbm}
\usepackage{mathspec}
\usepackage{amsmath, amsthm, amssymb}
\usepackage{mathtools}
\usepackage{latexsym}
\usepackage{ulem}
\usepackage{pdfpages}
%% \usepackage{mathenv}
%% \usepackage{bm}
\usepackage{textcomp}
\usepackage{icomma}
\usepackage{latexsym}
\usepackage{textcomp}
\usepackage{ulem}
\usepackage[T1]{fontspec}
\usepackage{enumitem}
\usepackage{pifont}
% \usepackage[a4paper]{geometry}
\usepackage{geometry}
\usepackage{hyperref}
% \usepackage[dvipsnames]{xcolor}
\usepackage{mathrsfs}
\usepackage{fancybox}
\usepackage{eucal}
\usepackage{dsfont}
\usepackage[most]{tcolorbox}
\usepackage{mathdots}
\usepackage{minted} %? used for code-integration
\usepackage{tkz-base}
\usepackage[linguistics]{forest}
\usepackage{amsfonts,amscd}
\usepackage{pgfplots}
\usepackage{subfig}
\usepackage{enumitem}
\usetikzlibrary{arrows.meta}

%% Pour changer les marges:
\setbeamersize{
  text margin left = 0.25cm, % normalement c'est 1 cm
  text margin right = 0.25cm % normalement c'est 1 cm
}

% Pour obtenir les nombres romain

\makeatletter
\newcommand*{\rom}[1]{\expandafter\@slowromancap\romannumeral #1@}
\makeatother

%% \newenvironment{changemargin}[2]{%
%%   \begin{list}{}{%
%%       \setlength{\topsep}{0pt}%
%%       \setlength{\leftmargin}{#1}%
%%       \setlength{\rightmargin}{#2}%
%%       \setlength{\listparindent}{\parindent}%
%%       \setlength{\itemindent}{\parindent}%
%%       \setlength{\parsep}{\parskip}%
%%     }%
%%   \item[]}{\end{list}
%% }

%%%%%%%%%%%%%%%%%%%%%%%%%%%%%%%%%%%%%

% Pour supprimer le "Figure" dans les captions
\usepackage[center]{caption}
\DeclareCaptionFormat{legend}{#3}
\captionsetup{format = legend}

\usepackage{cancel}

\usepackage{cleveref}

\usepackage{pgffor}

% Pour afficher du code
% Penser à \begin{frame}[fragile]
%%%%%%%%%%%%%%%%%%%%%%%%%%%%%%
\usepackage{listings}
\definecolor{darkWhite}{rgb}{0.94,0.94,0.94}

\lstset{
  aboveskip=2mm,
  belowskip=2mm,
  backgroundcolor=\color{darkWhite},
  basicstyle=\small\ttfamily,
  breakatwhitespace=false,
  breaklines=true,
  captionpos=b,
  commentstyle=\color{red},
  deletekeywords={...},
  %% escapeinside={\%}{\%},
  extendedchars=true,
  xleftmargin=16pt,
  framexleftmargin=6pt,
  framextopmargin=0pt,
  framexbottommargin=0pt,
  frame=tb,
  keepspaces=true,
  keywordstyle=\color{blue},
  language=caml,
  literate=
  {²}{{\textsuperscript{2}}}1
  {⁴}{{\textsuperscript{4}}}1
  {⁶}{{\textsuperscript{6}}}1
  {⁸}{{\textsuperscript{8}}}1
  {€}{{\euro{}}}1
  {é}{{\'e}}1
  {è}{{\`{e}}}1
  {ê}{{\^{e}}}1
  {ë}{{\¨{e}}}1
  {É}{{\'{E}}}1
  {Ê}{{\^{E}}}1
  {û}{{\^{u}}}1
  {ù}{{\`{u}}}1
  {â}{{\^{a}}}1
  {à}{{\`{a}}}1
  {á}{{\'{a}}}1
  {ã}{{\~{a}}}1
  {Á}{{\'{A}}}1
  {Â}{{\^{A}}}1
  {Ã}{{\~{A}}}1
  {ç}{{\c{c}}}1
  {Ç}{{\c{C}}}1
  {õ}{{\~{o}}}1
  {ó}{{\'{o}}}1
  {ô}{{\^{o}}}1
  {Õ}{{\~{O}}}1
  {Ó}{{\'{O}}}1
  {Ô}{{\^{O}}}1
  {î}{{\^{i}}}1
  {Î}{{\^{I}}}1
  {í}{{\'{i}}}1
  {Í}{{\~{Í}}}1,
  morekeywords={},
  numbers=left, framexleftmargin=16pt,
  numbersep=10pt,
  numberstyle=\tiny\color{black},
  rulecolor=\color{black},
  showspaces=false,
  showstringspaces=false,
  showtabs=false,
  stepnumber=1,
  stringstyle=\color{gray},
  tabsize=4,
  % Entourer la reponse de caml par $ ... $
  moredelim=[is][\color{green2}\ttfamily]{$}{$},
  %% title=\lstname,
}

%% %%% Pour faire centrer des itemize
%% \newcommand\centrer[1]{%
%%   \begin{center}
%%     \begin{minipage}[t]{\widthof{#1}}
%%       #1%
%%     \end{minipage}
%%   \end{center}
%% }


\newcommand{\py}[1]{
\begin{lstlisting}[language=python]
#1
\end{lstlisting}
}

\newcommand{\btt}[1]{\blue{\texttt{#1}}}
\newcommand{\gtt}[1]{\green{\texttt{#1}}}
\newcommand{\rtt}[1]{\red{\texttt{#1}}}
\newcommand{\stt}[1]{\small\texttt{#1}}
\newcommand{\ftt}[1]{\footnotesize\texttt{#1}}
%% \newcommand{\tt}[1]{\texttt{#1}}

\newcommand{\co}{[}
\newcommand{\cf}{]}



\newcommand{\cmark}{\ding{51}}%
\newcommand{\xmark}{\ding{55}}%

%%%%%%%%%%%%%%%%%%%%%%%%%%%%%%%%%%

\newcommand\itemref[1]{{\renewcommand{\insertenumlabel}{\ref{#1}}%
    \usebeamertemplate{enumerate item}}}

    %!
\usepackage{natbib}
% \usepackage{bibentry}
% \bibliographystyle{alpha}

%\setbeamertemplate{bibliography item}{[bonjour]}
%\setbeamertemplate{bibliography item}[text]

%% \usepackage{tabularx,cellspace}


% \usepackage[]{biblatex}
\usepackage[backend=biber,style=authortitle]{biblatex}
% \renewbibmacro{in:}{} %Pour supprimer le dans les refs
%% \bibliography{QEC}

\usepackage{xcolor}
\definecolor{green2}{RGB}{6,171,18}
\newcommand{\red}[1]{\textcolor{red}{#1}}
\newcommand{\blue}[1]{\textcolor{blue}{#1}}
\newcommand{\green}[1]{\textcolor{green2}{#1}}
\newcommand{\white}[1]{\textcolor{white}{#1}}
\newcommand{\black}[1]{\textcolor{black}{#1}}
\newcommand{\yellow}[1]{\textcolor{yellow}{#1}}
\newcommand{\magenta}[1]{\textcolor{magenta!70!red}{#1}}


\newcommand{\bN}{\mathbb{N}}
\newcommand{\bR}{\mathbb{R}}
\newcommand{\bZ}{\mathbb{Z}}



\newcommand{\cA}{\mathcal{A}}
\newcommand{\cB}{\mathcal{B}}
\newcommand{\cC}{\mathcal{C}}
\newcommand{\cD}{\mathcal{D}}
\newcommand{\cL}{\mathcal{L}}
\newcommand{\cO}{\mathcal{O}}
\newcommand{\cP}{\mathcal{P}}
\newcommand{\cV}{\mathcal{V}}
\newcommand{\cE}{\mathcal{E}}

\newcommand{\sV}{\mathscr{V}}
\newcommand{\sE}{\mathscr{E}}
\newcommand{\sA}{\mathscr{A}}
\newcommand{\sL}{\mathscr{L}}

\def\inte #1 #2 { [\![#1,#2]\!] }

\DeclareMathOperator{\ddroit}{d}

\renewcommand{\arraystretch}{1.4} %? For better matrix

%% Pour supprimer la barre de navigation
\beamertemplatenavigationsymbolsempty

\newlength{\hauteurFrameTOC}
\setlength{\hauteurFrameTOC}{0.65\textheight}
\newlength{\upTOC}
\setlength{\upTOC}{-10pt}


\AtBeginSection[]
{
  \begin{frame}<beamer>
    \frametitle{Sommaire}
    \vspace{\upTOC}
    \hfill
    \parbox[t]{.95\textwidth}{
      \begin{minipage}[c][\hauteurFrameTOC]{\textwidth}
        \tableofcontents[currentsection]
      \end{minipage}
    }
  \end{frame}
}


% Pour changer le stye des points itemize
\setbeamertemplate{itemize item}[triangle]

\newlength{\lengthleft}
\newlength{\lengthright}

%%%%%%%%%%%%%%%%%%%%%%%%%%%%%%%%%%%%%%%%%%%%%%%%%%%%%%%%%%%%%%%%%%%%%%%%%
% Pour faire des tableaux
\usepackage{array}
\usepackage{makecell}
%% \usepackage{diagbox}
%% \usepackage{float}


%% \newcolumntype{m}{>{$}c<{$}} % math-mode version of "l" column type

\newcolumntype{R}[1]{>{\raggedleft\arraybackslash }m{#1}}
\newcolumntype{L}[1]{>{\raggedright\arraybackslash }m{#1}<{}}
\newcolumntype{C}[1]{>{\centering\arraybackslash }m{#1}}


\newcolumntype{t}{>{\centering\small\ttfamily\arraybackslash }c}

\newcolumntype{E}[2]{>{\vspace*{#2}}C{#1}<{\vspace*{-1pt}\vspace*{#2}}}
%% \newcolumntype{D}[1]{>{\centering}p{#1}}


\newlength{\tailleHorizontale}
\newlength{\vspaceLength}

\newlength{\vspacetable}
\setlength{\vspacetable}{3pt}
%% \newcolumntype{A}[1]{>{\vspace{\vspacetable}\raggedleft\arraybackslash }m{#1}<{\vspace{\vspacetable}}}
%% \newcolumntype{L}[1]{>{\vspace{\vspacetable}\raggedright\arraybackslash }m{#1}<{\vspace{\vspacetable}}}
\newcolumntype{A}[1]{>{\vspace{\vspacetable}\centering\arraybackslash }m{#1}}
%%%%%%%%%%%%%%%%%%%%%%%%%%%%%%%%%%%%%%%%%%%%%%%%%%%%%%%%%%%%%%%%%%%%%%%%%


%%%%%%%%%%%%%%%%%%%%%%%%%%%%%%%%%%%%%%%%%%%%%%%%%%%%%%%%%%%%%%%%%%%%%%
%%%%%%%%%%%%%%%%%%%%%%%%%%%%%%%%%%%%%%%%%%%%%%%%%%%%%%%%%%%%%%%%%%%%%%
%%%%%%%%%%%%%%%%%%%%%%%%%%%%%%%%%%%%%%%%%%%%%%%%%%%%%%%%%%%%%%%%%%%%%%
%%%%%%%%%%%%%%%%%%%%%%%%%%%%%%%%%%%%%%%%%%%%%%%%%%%%%%%%%%%%%%%%%%%%%%
%%%%%%%%%%%%%%%%%%%%%%%%%%%%%%%%%%%%%%%%%%%%%%%%%%%%%%%%%%%%%%%%%%%%%%
%%% \newcommand


%%%%%%%%%%%%%%%%%%%%%%%%%%%%%%%%%%%%%%%%%%%%%%%%%%%%%%%%%%%%%%%%%%%%%%
%%%%%%%%%%%%%%%%%%%%%%%%%%%%%%%%%%%%%%%%%%%%%%%%%%%%%%%%%%%%%%%%%%%%%%
%%%%%%%%%%%%%%%%%%%%%%%%%%%%%%%%%%%%%%%%%%%%%%%%%%%%%%%%%%%%%%%%%%%%%%
%%%%%%%%%%%%%%%%%%%%%%%%%%%%%%%%%%%%%%%%%%%%%%%%%%%%%%%%%%%%%%%%%%%%%%
%%%%%%%%%%%%%%%%%%%%%%%%%%%%%%%%%%%%%%%%%%%%%%%%%%%%%%%%%%%%%%%%%%%%%%

\setbeamertemplate{headline}[default]

\defbeamertemplate*{footline}{infolines theme}
{
  \hbox{%
    \begin{beamercolorbox}[wd=.22\paperwidth,ht=2.25ex,dp=1ex,center]{author in head/foot}%
      TIPE 2021
    \end{beamercolorbox}%
    \begin{beamercolorbox}[wd=.70\paperwidth,ht=2.25ex,dp=1ex,center]{title in head/foot}%
      \insertshorttitle
    \end{beamercolorbox}%
    \begin{beamercolorbox}[wd=.08\paperwidth,ht=2.25ex,dp=1ex,center]{author in head/foot}%
      \insertframenumber/\inserttotalframenumber
  \end{beamercolorbox}  }%
}

% Supprime les ombres des block (ces trucs buguent aujourd'hui)
\setbeamertemplate{blocks}[rounded][shadow=false]


% Pour cacher des frames:
\newcounter{FrameNumberBeforeAppendix}
\newenvironment{annexes}{
  \setcounter{FrameNumberBeforeAppendix}{\value{framenumber}}
}{
  \setcounter{framenumber}{\value{FrameNumberBeforeAppendix}}
}

\usepackage{lipsum}


\renewcommand{\leq}{\leqslant}
\renewcommand{\geq}{\geqslant}




%%%%%%%%%%%%%%%%%%%%%%%%%%%%%%%%%%%%%%%%%%%%%%%%%%%%%%%%%%%%%%%%
%%% Pour donner des tests à faire
%%% Source: https://tex.stackexchange.com/questions/72902/new-command-with-variable-number-of-parameters
%%%%%%%%%%%%%%%%%%%%%%%%%%%%%%%%%%%%%%%%%%%%%%%%%%%%%%%%%%%%%%%%

\newcounter{nbParLigne}
\newcounter{nbSurLigne}


\newcommand{\aligntext}[1][,]{
  \def\separateur{#1}
  \aligntextRelay
}

\newcommand{\aligntextRelay}[3][.]{
  \def\separateurFinal{#1}
  \begin{align*}
    \alignText{#2}
    #3
    \stop
  \end{align*}
}

\makeatletter % we need to use kernel commands
\newcommand{\alignText}[2]{%
  \setcounter{nbParLigne}{#1}
  \@alignTextPremierTest{#2}
  \setcounter{nbSurLigne}{1}
  \@alignTexti
}
\newcommand\@alignTexti{\@ifnextchar\stop{\@alignTextEnd}{\@alignTextii}}
\newcommand\@alignTextii[1]{%
  \ifnum\value{nbParLigne}=\value{nbSurLigne}
  \@alignTextNvLigne{#1}
  \setcounter{nbSurLigne}{1}
  \else
  \@alignTextNvTest{#1}
  \addtocounter{nbSurLigne}{1}
  \fi
  \@alignTexti % restart the recursion
}
\newcommand\@alignTextPremierTest[1]{%
  & \text{#1}
}
\newcommand\@alignTextNvLigne[1]{%
  \separateur\\& \text{#1}
}
\newcommand\@alignTextNvTest[1]{%
  \separateur&& \text{#1}
}
\newcommand\@alignTextEnd[1]{% The argument is \stop
  \separateurFinal
}
\makeatother

%%%%%%%%%%%%%%%%%%%%%%%%%%%%%%%%%%%%%%%%%%%%%%%%%%%%%%%%%%%%%%%%
%%%%%%%%%%%%%%%%%%%%%%%%%%%%%%%%%%%%%%%%%%%%%%%%%%%%%%%%%%%%%%%%
%%%%%%%%%%%%%%%%%%%%%%%%%%%%%%%%%%%%%%%%%%%%%%%%%%%%%%%%%%%%%%%%
%%% TODO: faire un truc mieux que ce gros copier/coller

\newcommand{\aligntexttt}[1][,]{
  \def\separateur{#1}
  \aligntextttRelay
}

\newcommand{\aligntextttRelay}[3][.]{
  \def\separateurFinal{#1}
  \begin{align*}
    \alignTexttt{#2}
    #3
    \stop
  \end{align*}
}

\makeatletter % we need to use kernel commands
\newcommand{\alignTexttt}[2]{%
  \setcounter{nbParLigne}{#1}
  \@alignTextttPremierTest{#2}
  \setcounter{nbSurLigne}{1}
  \@alignTexttti
}
\newcommand\@alignTexttti{\@ifnextchar\stop{\@alignTextttEnd}{\@alignTextttii}}
\newcommand\@alignTextttii[1]{%
  \ifnum\value{nbParLigne}=\value{nbSurLigne}
  \@alignTextttNvLigne{#1}
  \setcounter{nbSurLigne}{1}
  \else
  \@alignTextttNvTest{#1}
  \addtocounter{nbSurLigne}{1}
  \fi
  \@alignTexttti % restart the recursion
}
\newcommand\@alignTextttPremierTest[1]{%
  & \texttt{#1}
}
\newcommand\@alignTextttNvLigne[1]{%
  \separateur\\& \texttt{#1}
}
\newcommand\@alignTextttNvTest[1]{%
  \separateur&& \texttt{#1}
}
\newcommand\@alignTextttEnd[1]{% The argument is \stop
  \separateurFinal
}
\makeatother

%%%%%%%%%%%%%%%%%%%%%%%%%%%%%%%%%%%%%%%%%%%%%%%%%%%%%%%%%%%%%%%%
%%%%%%%%%%%%%%%%%%%%%%%%%%%%%%%%%%%%%%%%%%%%%%%%%%%%%%%%%%%%%%%%
%%%%%%%%%%%%%%%%%%%%%%%%%%%%%%%%%%%%%%%%%%%%%%%%%%%%%%%%%%%%%%%%
%%% TODO: faire un truc mieux que ce gros copier/coller


\newcommand{\aligneq}[1][,]{
  \def\separateur{#1}
  \aligneqRelay
}

\newcommand{\aligneqRelay}[3][.]{
  \def\separateurFinal{#1}
  \begin{align*}
    \alignEq{#2}
    #3
    \stop
  \end{align*}
}

\makeatletter % we need to use kernel commands
\newcommand{\alignEq}[2]{%
  \setcounter{nbParLigne}{#1}
  \@alignEqPremierTest{#2}
  \setcounter{nbSurLigne}{1}
  \@alignEqi
}
\newcommand\@alignEqi{\@ifnextchar\stop{\@alignEqEnd}{\@alignEqii}}
\newcommand\@alignEqii[1]{%
  \ifnum\value{nbParLigne}=\value{nbSurLigne}
  \@alignEqNvLigne{#1}
  \setcounter{nbSurLigne}{1}
  \else
  \@alignEqNvTest{#1}
  \addtocounter{nbSurLigne}{1}
  \fi
  \@alignEqi % restart the recursion
}
\newcommand\@alignEqPremierTest[1]{%
  & #1
}
\newcommand\@alignEqNvLigne[1]{%
  \separateur\\& #1
}
\newcommand\@alignEqNvTest[1]{%
  \separateur&& #1
}
\newcommand\@alignEqEnd[1]{% The argument is \stop
  \separateurFinal
}
\makeatother

%%%%%%%%%%%%%%%%%%%%%%%%%%%%%%%%%%%%%%%%%%%%%%%%%%%%%%%%%%%%%%%%
%%%%%%%%%%%%%%%%%%%%%%%%%%%%%%%%%%%%%%%%%%%%%%%%%%%%%%%%%%%%%%%%
%%%%%%%%%%%%%%%%%%%%%%%%%%%%%%%%%%%%%%%%%%%%%%%%%%%%%%%%%%%%%%%%

% %? Nice boxes
% %%%%%%%%%%%%%
% \newcommand{\theocolor}{magenta}
% \newcommand{\theoremecolor}{blue}
% \newcommand{\lemmecolor}{green}
% \newcommand{\defcolor}{yellow}

% \definecolor{ShadowColor}{RGB}{30,150,190}

% \makeatletter
% \newcommand\Cshadowbox{\VerbBox\@Cshadowbox}
% \def\@Cshadowbox#1{%
%   \setbox\@fancybox\hbox{\fbox{#1}}%
%   \leavevmode\vbox{%
%     \offinterlineskip
%     \dimen@=\shadowsize
%     \advance\dimen@ .5\fboxrule
%     \hbox{\copy\@fancybox\kern.5\fboxrule\lower\shadowsize\hbox{%
%       \color{ShadowColor}\vrule \@height\ht\@fancybox \@depth\dp\@fancybox \@width\dimen@}}%
%     \vskip\dimexpr-\dimen@+0.5\fboxrule\relax
%     \moveright\shadowsize\vbox{%
%       \color{ShadowColor}\hrule \@width\wd\@fancybox \@height\dimen@}}}
% \makeatother


% \colorlet{ShadowColor}{\theocolor!70!black}
% \colorlet{ShadowColorTheoreme}{\theoremecolor!70!black}
% \colorlet{ShadowColorLemme}{\lemmecolor!70!black}
% \colorlet{ShadowColorDef}{\defcolor!70!black}

% \newenvironment{theo}{\vspace{10px}
%     \begin{tcolorbox}[colback=\theocolor!5!white, colframe=\theocolor!80!black, enhanced jigsaw,sharp corners, drop shadow=ShadowColor]
% }
% {\end{tcolorbox}
% \vspace{10px}
% }

% \newenvironment{theoreme}{\vspace{10px}
%     \begin{tcolorbox}[colback=\theoremecolor!5!white, colframe=\theoremecolor!80!black, enhanced jigsaw,sharp corners, drop shadow=ShadowColorTheoreme]
%     \textbf{Théorème:}
% }
% {\end{tcolorbox}
% \vspace{10px}
% }

% \newenvironment{lemme}{\vspace{10px}
%     \begin{tcolorbox}[colback=\lemmecolor!5!white, colframe=\lemmecolor!80!black, enhanced jigsaw,sharp corners, drop shadow=ShadowColorLemme]
%     \textbf{Lemme:}
% }
% {\end{tcolorbox}
% \vspace{10px}
% }

% \newenvironment{defi}{\vspace{10px}
%     \begin{tcolorbox}[colback=\defcolor!5!white, colframe=\defcolor!80!black, enhanced jigsaw,sharp corners, drop shadow=ShadowColorDef]
%     \textbf{Définition:}
% }
% {\end{tcolorbox}
% \vspace{10px}
% }
% %%%%%%%%%%%%%

% \setbeamertemplate{title page}
% {
%   \vbox{}
%   \vfill
%   \begin{centering}
%     \begin{theo}[sep=8pt,center]{title}
%       \usebeamerfont{title}\inserttitle
%     \end{theo}
%     \setbeamercolor{title}{bg=white,fg=structure}
%     % \begin{beamercolorbox}[sep=8pt,center]{title}
%     %   {\usebeamercolor[fg]{titlegraphic}\inserttitlegraphic\par}
%     %   \ifx\insertsubtitle\@empty%
%     %   \else%
%     %     \vskip0.25em%
%     %     {\usebeamerfont{subtitle}\usebeamercolor[fg]{subtitle}\insertsubtitle\par}%
%     %   \fi%
%     % \end{beamercolorbox}%
%     \vskip1em\par
%     \begin{beamercolorbox}[sep=8pt,center]{author}
%       \usebeamerfont{author}\insertauthor
%     \end{beamercolorbox}
%     \vskip-1em\par % change here
%     \begin{beamercolorbox}[sep=8pt,center]{institute}
%       \usebeamerfont{institute}\insertinstitute
%     \end{beamercolorbox}
%     \begin{beamercolorbox}[sep=8pt,center]{date}
%       \usebeamerfont{date}\insertdate
%     \end{beamercolorbox}\vskip0.5em
%   \end{centering}
%   \vfill
% }
% \makeatother
